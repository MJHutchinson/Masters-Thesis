\usepackage[pdftex,dvipsnames]{xcolor}

% ******************************************************************************
% ****************************** Custom Margin *********************************

% Add `custommargin' in the document class options to use this section
% Set {innerside margin / outerside margin / topmargin / bottom margin}  and
% other page dimensions
\ifsetCustomMargin
  \RequirePackage[left=25mm,right=25mm,top=25mm,bottom=25mm]{geometry}
  \setFancyHdr % To apply fancy header after geometry package is loaded
\fi

% Add spaces between paragraphs
%\setlength{\parskip}{0.5em}
% Ragged bottom avoids extra whitespaces between paragraphs
\raggedbottom
% To remove the excess top spacing for enumeration, list and description
%\usepackage{enumitem}
%\setlist[enumerate,itemize,description]{topsep=0em}

% *****************************************************************************
% ******************* Fonts (like different typewriter fonts etc.)*************

% Add `customfont' in the document class option to use this section

\ifsetCustomFont
  % Set your custom font here and use `customfont' in options. Leave empty to
  % load computer modern font (default LaTeX font).
  %\RequirePackage{helvet}

  % For use with XeLaTeX
  %  \setmainfont[
  %    Path              = ./libertine/opentype/,
  %    Extension         = .otf,
  %    UprightFont = LinLibertine_R,
  %    BoldFont = LinLibertine_RZ, % Linux Libertine O Regular Semibold
  %    ItalicFont = LinLibertine_RI,
  %    BoldItalicFont = LinLibertine_RZI, % Linux Libertine O Regular Semibold Italic
  %  ]
  %  {libertine}
  %  % load font from system font
  %  \newfontfamily\libertinesystemfont{Linux Libertine O}
\fi

% *****************************************************************************
% **************************** Custom Packages ********************************

% ************************* Algorithms and Pseudocode **************************

%\usepackage{algpseudocode}


% ********************Captions and Hyperreferencing / URL **********************

% Captions: This makes captions of figures use a boldfaced small font.
%\RequirePackage[small,bf]{caption}

\RequirePackage[labelsep=space,tableposition=top]{caption}
\renewcommand{\figurename}{Fig.} %to support older versions of captions.sty


% *************************** Graphics and figures *****************************

%\usepackage{rotating}
%\usepackage{wrapfig}

% Uncomment the following two lines to force Latex to place the figure.
% Use [H] when including graphics. Note 'H' instead of 'h'
%\usepackage{float}
%\restylefloat{figure}

% Subcaption package is also available in the sty folder you can use that by
% uncommenting the following line
% This is for people stuck with older versions of texlive
%\usepackage{sty/caption/subcaption}
\usepackage{subcaption}

% ********************************** Tables ************************************
\usepackage{booktabs} % For professional looking tables
\usepackage{multirow}

%\usepackage{multicol}
%\usepackage{longtable}
%\usepackage{tabularx}


% *********************************** SI Units *********************************
\usepackage{siunitx} % use this package module for SI units


% ******************************* Line Spacing *********************************

% Choose linespacing as appropriate. Default is one-half line spacing as per the
% University guidelines

% \doublespacing
% \onehalfspacing
% \singlespacing


% ************************ Formatting / Footnote *******************************

% Don't break enumeration (etc.) across pages in an ugly manner (default 10000)
%\clubpenalty=500
%\widowpenalty=500

%\usepackage[perpage]{footmisc} %Range of footnote options


% *****************************************************************************
% *************************** Bibliography  and References ********************

%\usepackage{cleveref} %Referencing without need to explicitly state fig /table

% Add `custombib' in the document class option to use this section

%\usepackage{natbib}
%\setcitestyle{authoryear,open={(},close={)}}
%
%\ifuseCustomBib
%   %\RequirePackage[round, sort, numbers, authoryear]{natbib} % CustomBib
%   \usepackage{natbib}
%   \setcitestyle{authoryear,open={(},close={)}}
   

% If you would like to use biblatex for your reference management, as opposed to the default `natbibpackage` pass the option `custombib` in the document class. Comment out the previous line to make sure you don't load the natbib package. Uncomment the following lines and specify the location of references.bib file

%\RequirePackage[backend=biber, style=numeric-comp, citestyle=numeric, sorting=nty, natbib=true]{biblatex}
%\addbibresource{References/references} %Location of references.bib only for biblatex, Do not omit the .bib extension from the filename.
%
%\fi

% changes the default name `Bibliography` -> `References'
\renewcommand{\bibname}{References}


% ******************************************************************************
% ************************* User Defined Commands ******************************
% ******************************************************************************

% *********** To change the name of Table of Contents / LOF and LOT ************

%\renewcommand{\contentsname}{My Table of Contents}
%\renewcommand{\listfigurename}{My List of Figures}
%\renewcommand{\listtablename}{My List of Tables}


% ********************** TOC depth and numbering depth *************************

\setcounter{secnumdepth}{2}
\setcounter{tocdepth}{2}


% ******************************* Nomenclature *********************************

% To change the name of the Nomenclature section, uncomment the following line

%\renewcommand{\nomname}{Symbols}


% ********************************* Appendix ***********************************

% The default value of both \appendixtocname and \appendixpagename is `Appendices'. These names can all be changed via:

%\renewcommand{\appendixtocname}{List of appendices}
%\renewcommand{\appendixname}{Appndx}

% *********************** Configure Draft Mode **********************************

% Uncomment to disable figures in `draft'
%\setkeys{Gin}{draft=true}  % set draft to false to enable figures in `draft'

% These options are active only during the draft mode
% Default text is "Draft"
%\SetDraftText{DRAFT}

% Default Watermark location is top. Location (top/bottom)
%\SetDraftWMPosition{bottom}

% Draft Version - default is v1.0
\SetDraftVersion{v3.0}

% Draft Text grayscale value (should be between 0-black and 1-white)
% Default value is 0.75
%\SetDraftGrayScale{0.8}


% ******************************** Todo Notes **********************************
%% Uncomment the following lines to have todonotes.


\usepackage{xargs}

\usepackage[colorinlistoftodos,prependcaption,textsize=tiny]{todonotes}

\ifsetDraft
	%\usepackage[pdftex,dvipsnames]{xcolor}
	\newcommandx{\unsure}[2][1=]{\todo[linecolor=red,backgroundcolor=red!25,bordercolor=red,#1,inline]{#2}}
	\newcommandx{\change}[2][1=]{\todo[linecolor=blue,backgroundcolor=blue!25,bordercolor=blue,#1,inline]{#2}}
	\newcommandx{\info}[2][1=]{\todo[linecolor=OliveGreen,backgroundcolor=OliveGreen!25,bordercolor=OliveGreen,#1,inline]{#2}}
	\newcommandx{\improvement}[2][1=]{\todo[linecolor=Plum,backgroundcolor=Plum!25,bordercolor=Plum,#1,inline]{#2}}
	\newcommandx{\thiswillnotshow}[2][1=]{\todo[disable,#1,inline]{#2}}
	\newcommand{\mynote}[1]{\todo[inline]{#1}}
\else
	\newcommand{\mynote}[1]{}
	\newcommandx{\unsure}[2][1=]{}
	\newcommandx{\change}[2][1=]{}
	\newcommandx{\info}[2][1=]{}
	\newcommandx{\improvement}[2][1=]{}
	\newcommandx{\thiswillnotshow}[2][1=]{}
	%\newcommand{\listoftodos}{}
\fi

% Example todo: \mynote{Hey! I have a note}

% *****************************************************************************
% ******************* Better enumeration my MB*************
\usepackage{enumitem}

% My stuff added
\usepackage{parskip}
\setlength{\parindent}{1em}
\usepackage{amssymb}
\usepackage{lipsum}
\usepackage{layouts}
\usepackage{placeins}
\usepackage{pdflscape}
\usepackage{multirow}
\usepackage[ruled,vlined,linesnumbered]{algorithm2e}
\usepackage{array,ragged2e}
\usepackage{tikz}
\usepackage{afterpage}

%\usepackage[bitstream-charter]{mathdesign}
%\usepackage{charter} 

%\setlength{\parskip}{0em}

\usepackage{titlesec}
\titleformat{\chapter}[display]   
{\normalfont\huge\bfseries}{\chaptertitlename\ \thechapter}{20pt}{\Huge}   
\titlespacing*{\chapter}{0pt}{0pt}{40pt}

%*********** Table definitions ***********

\newcolumntype{L}[1]{>{\raggedright\let\newline\\\arraybackslash\hspace{0pt}}m{#1}}
\newcolumntype{C}[1]{>{\centering\let\newline\\\arraybackslash\hspace{0pt}}m{#1}}
\newcolumntype{R}[1]{>{\raggedleft\let\newline\\\arraybackslash\hspace{0pt}}m{#1}}

%*********** Math definitions ***********

\DeclareMathOperator{\p}{p}
\DeclareMathOperator{\Tr}{Tr}
\DeclareMathOperator{\InvGamma}{Inv-Gamma}
\DeclareMathOperator{\ELBO}{ELBO}
\DeclareMathOperator{\GP}{\mathcal{GP}}
\DeclareMathOperator{\PI}{PI}
\DeclareMathOperator{\EI}{EI}
\DeclareMathOperator{\UCB}{UCB}
\DeclareMathOperator{\BNN}{BNN}

\newcommand{\D}{\mathcal{D}}
\newcommand{\E}{\mathbb{E}}
\newcommand{\R}{\mathbb{R}}
\newcommand{\DKL}[2]{\mathcal{D}_{KL} \left( #1 \| #2 \right) }

% define datasets
\def\bostonvar{bostonHousing}
\def\bostonname{Boston Housing}

\def\concretevar{concrete}
\def\concretename{Concrete}

\def\energyvar{energy}
\def\energyname{Energy}

\def\kinvar{kin8nm}
\def\kinname{Kinematics 8nm}

\def\mnistvar{mnist}
\def\mnistname{MINST }

\def\navalnvar{naval-propulsion-plant}
\def\navalnname{Naval Propulsion Plant}

\def\powervar{power-plant}
\def\powername{Power Plant}

\def\proteinvar{protein-tertiary-structure}
\def\proteinname{Protein Tertiary Structure}

\def\winevar{wine-quality-red}
\def\winename{Wine Quality}

\def\yachtnvar{yacht}
\def\yachtname{Yacht}


